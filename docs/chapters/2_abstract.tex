\chapter{Abstract}
\label{abstrakt}
Pod nadpisem Abstrakt je uvedeno shrnutí práce zabírající prostor maximálně 10 řádků. Z~dobrého abstraktu by mělo být i přes jeho malý rozsah patrné, jaký problém se řešil, jaký přístup k jeho řešení byl v práci použit a jakých výsledků bylo dosaženo. Účelem abstraktu je, aby potenciální čtenář práce již po přečtení abstraktu věděl, zda v práci najde to, co hledá \cite{fitWeb}. Zbytek této kapitoly byl převzat z blogu prof. Herouta \cite{Herout}.
\bigskip

\noindent Za prvé – na abstraktu záleží. Za druhé – není těžké ho napsat. Pojďme na to.

\subsection*{K čemu je abstrakt}
Abstrakt slouží k \bf vyhledávání\rm, společně s názvem dané vědecké práce a seznamem klíčových slov. Tyto části (snad s výjimkou názvu) nejsou přímo součástí textu a nečeká se, že někdo, kdo by zasedl ke čtení dané vědecké práce, bude číst je. To, že práci čte, znamená, že už se dostal za fázi čtení abstraktu. Abstrakt mu slouží ve chvíli, kdy se ještě rozhoduje, \bf zda vůbec \rm text číst.

Když někdo tam venku hledá odpověď na svůj problém, zadá knihovnici nebo dnes spíše vyhledávacímu serveru klíčová slova, která se jeho potíží dotýkají. Na základě shody těchto klíčových slov a klíčových slov uvedených autory dostane seznam názvů prací, které by mu mohly nabízet řešení. Dobře sestavený název práce badateli pomůže vytipovat takové texty, které by mohly mít vztah k jeho problému a mohl by se zajímat o jejich přečtení.

A tady právě přichází na scénu abstrakt. Badatel si čte abstrakt vytipovaných prací a~rozhoduje se, zda práci skutečně chce číst, nebo jestli se v tomto případě jeho filtr založený pouze na jednořádkovém nadpisu zmýlil.

V tuto chvíli obvykle ještě nemá stažené nějaké PDF s celým textem, natož aby měl v ruce vytištěný fascikl. Abstrakty jsou určeny k tomu, aby byly \bf mimo text\rm , aby ležely na serverech agregujících vědecké texty. Proto první pravidlo je, že abstrakt musí fungovat samostatně -- pokud obsahuje odkazy do literatury nebo se odvolává na text (\uv{Výkonnost metody je přehledně shrnuta na straně 51.}), nedělá badateli dobrou službu, což badatel ocení tím, že si o autoru nepomyslí nic hezkého, práci si nepřečte a autora neocituje.

\subsection*{Kdy a jak psát Abstrakt}
Může dávat smysl psát abstrakt na závěr celého psaní -- jako shrnutí a skutečné anotování sepsaného díla. Já jsem vyznavačem opačného přístupu -- abstrakt píšu na samém začátku. Když píšu vědecký článek, začínám sepsáním velkého počtu klíčových slov, jež se textu dotýkají. Bývá jich více, než potom uvedu jako ona charakteristická klíčová slova používaná k indexování. Ujasňuji si tím prostor, kde se článek pohybuje -- o čem je třeba hovořit, co je v textu podstatné, čeho se dotýká. Hned po ujasnění klíčových slov formuluji nadpis a~právě abstrakt.

Považuji za mimořádně užitečné ujasnit si právě ony čtyři části abstraktu -- Jaký problém se řeší? Jaké řešení práce nabízí? Jaké jsou přesně výsledky? Jaký je jejich význam? Když je toto jasné, text se píše skoro sám. Pokud toto má být nejasné, jak u všech všudy je možné vůbec dát dohromady smysluplnou větu v samém textu?

\subsection*{Doporučená struktura abstraktu}
Abstrakt vědecké práce se může skládat ze čtyř částí a pak být opravdu užitečný. Každá část se bude skládat z nějakých dvou, tří vět, někdy postačí jedna.

V byznysu se vžil slovesný útvar \uv{elevator pitch} -- představení ve výtahu. Ne náhodou jeho struktura připomíná právě doporučovanou strukturu abstraktu. Opravdu, autor odborného textu má do abstraktu napsat právě to, co by říkal o své práci, kdyby na to měl nejvýše dvě minuty a nemohl použít žádných slajdů, obrázků, textu. O čem by tedy měl mluvit?

\paragraph{První část -- Jaký se řeší problém? Jaké je téma? Jaký je cíl textu?}
\begin{itemize}
    \item{Tato práce řeší.}
    \item{Cílem této práce je.}
    \item{Zaměřil jsem se na.}
\end{itemize}
Nepatří sem úvodní pohádky charakteristické pro špatný odborný sloh: \uv{Naše poslední pětiletka staví před nás nové a smělé cíle}, \uv{S rozvojem výpočetní techniky a zejména zobrazovacích zařízení je stále důležitější \ldots} Ty nepatří do dobrého textu nikam, ale do~abstraktu tím méně. Pokud dokážete vyjádřit účel svého textu v jedné větě o pár slovech, udělejte to a nepřidávejte nic navíc. Stručnější zde vždy znamená lepší.

\paragraph{Druhá část -- Jak je problém vyřešen? Cíl naplněn?}
\begin{itemize}
    \item{Zvolený problém jsem vyřešil pomocí toho a toho.}
    \item{V řešení bylo použito metody té, postupu toho a analýzy oné.}
    \item{Práce představuje algoritmus takový, který.}
    \item{Data jsem zpracovával pomocí těch a těchto nástrojů a provedl vyhodnocení takové.}
    \item{Podstatou našeho algoritmu je.}
\end{itemize}

Pokud je podstatou sepisovaného odborného textu nová metodologie (= \uv{jak něco dělat}), patří přesně sem její popis. Pokud se představované řešení skládá ze tří částí, pravděpodobně v této části abstraktu budou tři věty, z nichž každá se bude věnovat jedné části řešení. Dobrý abstrakt v této části bude upřímný a přesný -- nebude si schovávat \uv{odhalování svých tajemství} až do textu. Vágní formulace podstaty řešení v abstraktu obvykle znamená, že autoři buď neumí psát a nebo vlastně nemají o čem -- ani jedno není zrovna výzva ke stažení a čtení mnoha stran textu.

\paragraph{Třetí část -- Jaké jsou konkrétní výsledky? Jak dobře je problém vyřešen?}
\begin{itemize}
    \item{Podařilo se dosáhnout úspěšnosti 87,3\,\%.}
    \item{V práci jsme vytvořili systém, který.}
    \item{Vytvořené řešení poskytuje ty a ty možnosti.}
    \item{Provedeným výzkumem jsme zjistili, že.}
\end{itemize}

Není špatným zvykem uvést v této části konkrétní číslo -- \uv{existující metodu XY jsme zrychlili pětkrát}. Pokud přínos práce není možné shrnout do dvou nebo tří vět, někde je něco velmi špatně a celý text pravděpodobně nestojí za psaní.

\paragraph{Čtvrtá část -- Takže co? Čím je to užitečné vědě a čtenáři?}
\begin{itemize}
    \item{Přínosem této práce je.}
    \item{Hlavním zjištěním je.}
    \item{Hlavním výsledkem je.}
    \item{Na základě zjištěných údajů je možné.}
    \item{Výsledky této práce umožňují.}
\end{itemize}

Při psaní vědeckých článků já sám obvykle bojuji s podobností části třetí a čtvrté. Vskutku, obě hovoří o tom, co jsou výsledky a přínosy textu. Účelem třetí části je jmenovitě a konkrétně jmenovat dosažené výsledky, úkolem části čtvrté je interpretovat jejich význam. Asi ničemu nevadí, když tato dvě sdělení do jisté míry splynou a část třetí a čtvrtá nejen že nemají každá vlastní odstavec, ale prolínají se dokonce ve společných větách.

\paragraph{Nultá část -- O co jde? Kde jsme?}
\begin{itemize}
    \item{Práce je řešena v kontextu tom a tom.}
    \item{Nauka ta a ta se zabývá studiem toho a toho.}
    \item{Stavíme na těchto a oněch nedávných pokrocích v naší oblasti.}
\end{itemize}

Někdy je nutné na sám začátek abstraktu vložit kratičké uvedení kontextu, ve kterém se~celá záležitost vlastně odehrává. Může to být přínosné~u vskutku obskurního a esoterického oboru, který leží stranou hlavního proudu. Obvykle tato část ovšem nebývá nutná a~věty v~ní obsažené bývají prototypy ohavné, rádobyodborné vaty. Je dobrou praxí zapomenout, že se tato část v abstraktu může vyskytovat. Když někdo, kdo je odborníkem v~oboru práce, přece po přečtení abstraktu zavrtí hlavou: \uv{Vůbec nevím, o čem tady můžete psát,} pouze tehdy je vhodné vložit nějaké věty s uvedením kontextu.

\subsection*{Inovace není Ignorance}

Popisuji v tomto textu jakýsi obecný model obecné diplomky. Ještě ke všemu se na začátku zaklínám, že to je můj názor a vkus a jsem zvědavý na názory a vkusy alternativní (což jsem!). Každý diplomant (Mgr. i Bc.) přitom cítí, že jeho diplomka je speciální a výjimečná. Tudíž se nebude držet nějakého schématu, které slouží pro běžné a průměrné diplomky -- tj. pro ty ostatní. Setkávám se s dobrými důvody, proč se od výše naznačeného schématu odchýlit a každoročně některým studentům odchýlení od schématu sám doporučuji. Vskutku, každá diplomka je jedinečná a zvláštní. Kdyby ne, nemusely by se psát, stačilo by je kopírovat. Ovšem vždycky před tím, než vybočíte ze standardního a kanonického způsobu organizování odborného textu, dejte si tu práci ho poznat, pochopit a zvládnout. Způsob vědecké práce, strukturování odborného textu, nebo třeba citování pramenů, může vypadat rigidně a neohrabaně, je to ale zatím ten nejlepší způsob, který jsme jako lidstvo dokázali vymyslet. Pokud ho ovládnete, pochopíte jeho výhody a nevýhody a inovujete ho, je to v pořádku a jste vítáni. Pokud se jím odmítnete zabývat, pravděpodobně neprovedete hodnotnou inovaci, ale vytvoříte \uv{paskvil}.
